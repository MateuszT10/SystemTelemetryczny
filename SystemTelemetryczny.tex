\documentclass[12pt]{report}
\usepackage[T1]{fontenc}
\usepackage[utf8]{inputenc}
\usepackage{graphicx}
\usepackage{amsmath,amssymb}
\usepackage{helvet}
\renewcommand{\familydefault}{\sfdefault}


\usepackage[polish]{babel}
\usepackage{geometry}
\newgeometry{tmargin=2.5cm, bmargin=2.5cm, lmargin=3cm, rmargin=2.5cm}

\renewcommand{\chaptername}{Rozdział}
\renewcommand{\contentsname}{Spis treści}
\renewcommand{\figurename}{Rys.}
\renewcommand{\tablename}{Tab.}
\renewcommand{\listfigurename}{Spis rysunków}
\renewcommand{\listtablename}{Spis tabel}
\renewcommand{\bibname}{Bibliografia}

\pagestyle{plain}

%\setlength{\textwidth}{14cm}
%\setlength{\textheight}{20cm}

\newtheorem{definition}{Definicja} % przykład nowego środowiska 
\newtheorem{example}{Przykład}[chapter] % przykład nowego środowiska 
\newtheorem{corollary}{Wniosek}[chapter] % przykład nowego środowiska 



\begin{document}
Tutaj bedzie strona tytulowa ;)
\tableofcontents	% generuje spis treści ze stronami !!!

\chapter{Wstęp} \label{rozdz.wstep} 


\chapter{Cel i zakres pracy} \label{etykietarozdzialu2}


\chapter{Teoria}


\chapter{Oprogramowanie}


\chapter{Weryfikacja działania}
%Błedy pomiaru
\chapter{Podsumowanie}
\addcontentsline{toc}{chapter}{Załączniki} 
\chapter*{Załączniki}
\begin{enumerate}
\item Załącznik nr 1
\item Załącznik nr 2
\item Załącznik nr 3
\end{enumerate}


\end{document}
